\documentclass[11pt,a4paper]{article}
\usepackage[utf8x]{inputenc}
\usepackage[T1]{fontenc}
%\usepackage{gentium}
\usepackage{mathptmx} % Use Times Font

\usepackage{graphicx} % Required for including pictures
\usepackage{hyperref} % Format links for pdf
\usepackage[british]{babel} % Multilingual bibliographies
\usepackage[square,numbers]{natbib}
\setlength{\bibsep}{0.0pt}
\usepackage{booktabs} % Used so that tables generated by pandas
                      % to_latex() work correctly

\frenchspacing % No double spacing between sentences
\usepackage[margin=1in]{geometry}

\usepackage[all]{nowidow} % Tries to remove widows
\usepackage[protrusion=true,expansion=true]{microtype} % Improves typography, load after fontpackage is selected

\usepackage{lipsum} % Used for inserting dummy 'Lorem ipsum' text into the template

\title{FDS Report}
\author{B133210}

\begin{document}

\maketitle

%% INSTRUCTIONS:
%%
%% 1. Edit this file either using:
%%
%%    a. Overleaf professional, a collaborative LaTeX editor. See
%%       https://www.overleaf.com/edu/edinburgh for documentation. Create an
%%       empty document, and copy the files in this directory to it.
%%
%%    b. A LaTeX editor on your PC.
%% 
%% 2. Follow steps 1 to 25 in the commented-out sections of the
%%    document.
%%
%% 3. Please keep the section and paragraph headings as they are.
%%    Delete the sections indicated “DELETE...” before submission.

\section{Introduction}


\paragraph{Background}

% Open McPherson (2017) "Glasgow's public cycle hire scheme: analysis
% of usage between July 2014 and June 2016"
% https://www.gcph.co.uk/assets/0000/6010/Next_bike_hire_data.pdf. We
% have provided a back-up of this file in the top level of this
% repository. Read pages 2-4 and pages 16-18 of McPherson (2017) now.

% 1. Write a paragraph below that gives a  short description of the
% the Glasgow and Edinburgh bike hire schemes, and give one or more
% reasons why they should interest the reader. You may wish to write
% this section after you have done most of the work.

\paragraph{The Glasgow public hire scheme}

% 2. Write one paragraph summarising three key points of the analysis
% of McPherson (2017) "Glasgow's public cycle hire scheme: analysis of
% usage between July 2014 and June 2016"
% https://www.gcph.co.uk/assets/0000/6010/Next_bike_hire_data.pdf.

DELETE THIS PARAGRAPH BEFORE SUBMISSION: Remember to always to cite
information that has come from other sources, to avoid plagiarism,
like this: \cite{wiki:plagarism}. The file fds-resit.bib contains some
examples, including one useful for this paper \cite{McPherson2017}.

% 3. In one or two paragraphs, evaluate McPherson (2017). This means
% identifying strengths and weaknesses, explaining why they are
% strengths and weaknesses, and suggesting what could have been done
% differently. For example, areas for improvement could include the
% informativeness of the visualisations, or descriptive or inferential
% statistics.

\paragraph{Objectives}

% 4. In one paragraph, state what the objectives of this study are
% (you will need to write this paragraph after you have written the
% rest of the report).

\section{Data}

\paragraph{Data provenance}

% 5. Go to https://edinburghcyclehire.com/open-data/historical
% (We have also provided a backup copy in the directory
% "datasets/edinburghcyclehire.com").
% 
% In this document describe the provenance of the dataset:
% - Cite the dataset and describe it in one sentence  (see
%   fds-resit.bib to help with citing).
% - Identify which license it was released under.
% - Explain why it is valid for us to adapt the data for this assignment.
% - Comment on any potential ethical issues using this data.

% We have downloaded hourly weather data from 2019 from the Gogarbank
% weather station from the MIDAS-open dataset, and put it in the
% "datasets" directory. We have adapted the data to make it easy to
% read in for this assignment. 

% 6. Look at the description of the dataset online:
% https://catalogue.ceda.ac.uk/uuid/8d85f664fc614ba0a28af3a2d7ef4533
% (We have also provided a backup copy as a PDF in the directory
% "datasets").
%
% In this document describe the provenance of the dataset:
% - Cite the dataset and describe it in one sentence (see
%   fds-resit.bib to help with citing)
% - Identify which license it was released under.
% - Explain why it is valid for us to adapt the data for this
%   assignment.

\paragraph{Data description}

% Now open a Jupyter notebook or python file.
%
% In the Jupyter notebook, download all the bike data files for
% 2019 and combine them into one DataFrame called "bike".
%
% Find the data dictionary ("Variables" section on
% https://edinburghcyclehire.com/open-data/historical) to help with
% the following question.

% 7. In this document, describe the data by using the data dictionary
% and your Jupyter notebook to give answers to the following
% questions:
%
% a. How many columns does the DataFrame contain?
The DataFrame contains 13 columns. 


% b. What are the types and meanings of the columns?  You can choose

% only to mention the ones that you will use.

% c. How many rows does the DataFrame contain?
The Dataframe contains 124649 rows.

%
% You may use a table for part (a). See the end of this document for
% sample LaTeX code to produce a table. If quoting verbatim from the
% Just Eat website, you should indicate this. Note that in LaTeX you
% need to put a backslash before an underscore: i.e. "\_" instead of
% "_".
%

% 8. Report below the mean, median, standard deviation, minimum and
% maximum of the "duration" column. Compare your values with those of
% McPherson (2017). Interpret what these values mean -- you may refer to
% the discussion of McPherson (2017) for potential explanations.
\begin{itemize}
\item  Mean duration :           1570.5535784482827
\item Standard deviation:     5036.690271179312
\item Median:                          912.0
\item Minimum duration:      61
\item Maximum duration:    688832
\item 
\end{itemize}


% Read the file
% "datasets/midas-open_uk-hourly-weather-obs_dv-202007_midlothian-in-lothian-region_19260_edinburgh-gogarbank_qcv-1_2019_adapted.csv" into a DataFrame called "weather"

% 9. Look at the data dictionary
% (datasets/midas-open_uk-hourly-weather-obs_dv-202007_data_dictionary.csv)
% to help with your descriptions. Below describe the data by giving
% answers to the following questions:
%
% a. How many columns does the DataFrame contain? 
The DataFrame 'weather' contains 3 columns.

% b. What are the types and meanings of the columns? 
% c. How many rows does the DataFrame contain. Explain why this number
%    is or is not appropriate a year's worth of hourly weather data.
The DataFrame contains 8760 rows.


% 10. Report summary statistics of the columns below. State if the
% values seem reasonable or not, and explain why.

\paragraph{Data processing}

% In the bike dataset, convert the start and end times to be a
% python datetimes object, using the pandas function pd.to_datetime().

% In the weather dataset, convert the observation time to be a
% python datetime object, using the pandas function pd.to_datetime().

% 11. Create a DataFrame containing a histogram of the counts of the
% bike data in the hours corresponding to the weather observations.
% For 1 point, you can use the following code ("weather" is the
% weather DataFrame you created in Questions 7 and 11 and "bike" is
% the bike DataFrame you created in Questions 8 and 12).

% bike['hour'] = pd.cut(bike['started_at'], bins=weather.ob_time, labels=weather.ob_time.drop(index=len(weather.ob_time) - 1))
% bike['hour'] = pd.to_datetime(bike['hour'])
% bike_count = bike[['started_at', 'hour']].groupby('hour', as_index=False).count()
% bike_count.rename(columns={'started_at': 'count'}, inplace=True)

% For up to 3 points, if you have a better way of doing this, explain
% that you wrote your own code and we will look at your notebook. You
% will only gain the points if the code and the following results
% appear to be correct.

% You should now have three data frames:
% - weather - the weather data
% - bike - the original bike data
% - bike_count - counts of bike hires each hour

% 12. In this document, describe the data processing you have
% undertaken in this section. Report the number of rows in the
% bike_count DataFrame, and explain why this number is appropriate.
There are 8076 rows in. the bike_count DataFrame.

\section{Exploration and  analysis}

\paragraph{Temperature and bike use in Edinburgh}

% 13. In the Jupyter notebook plot the air temperature versus time and
% the number of bike rentals per hour above each other. Save the
% figure and include it in this document; see example LaTeX code at
% end of document for how to include and caption a figure.


% 14. Describe what you see in the figure. What does the relationship
% between the number of bike rentals and temperature look like? Are
% there any anomalies in the data that should be investigated?
Both graphs have peaks around May time.
Generally, as the air temperature gets warmer the number of bike rentals increase.

% 15. Read Section 2.4 of McPherson (2017). Describe what Figure 12 of
% McPherson (2017) is being used to investigate. Identify the method
% McPherson (2017) used to produce their Figure 12 and how they
% applied the method. Interpret the equation shown in their Figure 12
% and the metric shown there.

% 16. In the Jupyter notebook, do the following:
%
% a. Merge the bike count DataFrame and the weather DataFrame so that
%    each row contains observations of weather and bike counts found in
%    that hour.

% b. Create a scatter plot of the bike count on the y-axis and the
%    air temperature on the x-axis. 
done
%
% c. Use a function of your choosing in the statsmodels package to
%    fit a model like McPherson's to the data in the scatter plot.
% 
??
% d. Produce a visualisation that includes the scatter plot and the
%    results of the model fitted in (c), and include it in this
%    document. Identify the function you chose in (c) to create the
%    figure in the caption.

% 17. In this document, describe the figure you have included.
%     Report the coefficients you found in 16c and confidence intervals
%     for the coefficients. Describe the meaning of the coefficients
%     that you found in 16c.  Discuss what can be
%     concluded about the relationship between bike use and temperature.
%     Compare what you find with what McPherson found about the
%     relationship between temperature and bike use.

\paragraph{Further discussion and exploration}

% 18. Choose EITHER option A or option B
% 
% A. Discuss whether the method McPherson (2017) has applied is
% appropriate for the data. What might you do differently? Try
% applying your idea, showing a visualisation and interpreting the
% results.
%
% B. Pick ONE of Figures 3, 4, 5 or 6 in McPherson (2017). Create the
% the equivalent visualisation for the Edinburgh bike data. You don't
% have to make the visualisation look exactly the same; you can
% experiment with different types of plots and try to convey more
% information. Include the visualisation in this document, giving it
% an appropriate caption (see LaTeX code below for how to do this).
% Explain any differences in the way you have presented your data from
% how McPherson presents the same plot. Interpret what your
% visualisation shows. Compare your visualisation with McPherson's
% visualisation. Are the patterns of bike usage in Glasgow and
% Edinburgh similar or different?


\section{Discussion and conclusions}


\paragraph{Summary of findings}

% 19. Summarise your findings in a few sentences.

\paragraph{Evaluation of own work: strengths and limitations}

% 20. In one paragraph, list any strengths and limitations in your own
% work.

\paragraph{Improvements and extensions}

% 21. In one paragraph, identify ideas for future work. This could
% include new questions to address or improving your analysis.


\section{EXAMPLES OF HOW TO INCLUDE FIGURES AND TABLES}

\textbf{DELETE THIS WHOLE SECTION BEFORE SUBMISSION}

% 't' means "try to position at the top of the page"
\begin{figure}[t]
  \centering
  \includegraphics{example1}
  \caption{DELETE OR REPLACE THIS FIGURE AND CAPTION BEFORE
    SUBMISSION: Demonstration figure. This caption explains more about
    the figure. Note that the font size of the labels in the plot is
    9pt, which is obtained by the settings as shown in the Jupyter
    notebook \texttt{example-figure.ipynb}.}
  \label{fds-project-template:fig:example1}
\end{figure}


% 'b' means "try to position at the bottom of the page"
\begin{table}[b]
  \caption{DELETE THIS TABLE BEFORE SUBMISSION: Excerpt from Scottish Index of Multiple Deprivation, 2016 edition.
    \url{https://simd.scot}. You may put more information in the caption.}
  \label{tab:example1}
\begin{tabular}{lrrrrrrr}
\hline\hline
\textbf{Location}&\textbf{Employ-}&\textbf{Illness}&\textbf{Attain-}&\textbf{Drive}  &\textbf{Drive}    &\textbf{Crime}&\dots\\
                 &\textbf{ment}   &                &\textbf{ment}   &\textbf{Primary}&\textbf{Secondary}&              &\\
\hline
\textbf{Macduff}&$10$&$ 95$&$5.3$&$1.5$&$6.6$&$249$&\dots\tabularnewline
\textbf{Kemnay}&$ 3$&$ 40$&$5.3$&$2.4$&$2.4$&$168$&\dots\tabularnewline
\textbf{Hilton}&$ 0$&$ 10$&$6.3$&$2.2$&$3.0$&$144$&\dots\tabularnewline
\textbf{Ruchill}&$ 8$&$130$&$4.9$&$1.7$&$5.6$&$318$&\dots\tabularnewline
\textbf{Belmont}&$ 2$&$ 50$&$6.1$&$3.1$&$3.2$&$129$&\dots\tabularnewline
\dots&\dots&\dots&\dots&\dots&\dots&\dots&\dots\tabularnewline
\hline
\end{tabular}
\end{table}

Here is how to refer to visualisations (for example
Figure~\ref{fds-project-template:fig:example1}) and tables (for
example Table~\ref{tab:example1}). Please make sure that all figures
and tables are referred to in the text, as demonstrated in this bullet
point.

You can use equations like this:
\begin{equation}
  \label{fds-project-template:eq:1}
  \overline{x} = \sum_{i=1}^n x_i
\end{equation}
or maths inline: $E=mc^2$. However, you do not need to reexplain techniques that you have learned in the course -- assume the reader understands linear regression, logistic regression K-nearest neighbours etc.  Remember to explain any symbols use, e.g.~``$n$ is the number of data points and $x_i$ is the value of the $i$th data point.''.


\bibliographystyle{unsrtnat}
\bibliography{fds-resit}
\end{document}

